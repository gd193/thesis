\section{Convolution and the Convolutional Theorem}\label{sec:conv}

Having established the basic properties of the Fourier transform, I will now introduce the convolution operation and its important properties.

\begin{definition}[Convolution]
    Let $f, g \in L^1(\mathbb{R}^d)$ be integrable functions. The convolution of $f$ and $g$, written as $(f \ast g)$, is defined as
    \begin{equation}
        (f \ast g)(t) := \int_{\mathbb{R^d}} f(\tau) g(t - \tau) d^d \tau
    \end{equation}
    \label{eq:conv_def}
\end{definition}
The convolution operation is also defined for discrete functions.

\begin{definition}[Discrete Convolution]
    Let $f, g$ be complex-valued functions defined on $\mathbb{Z}$. The convolution of $f$ and $g$ is defined as
    \begin{equation}
        (f \ast g)[n] := \sum_{m=-\infty}^{\infty} f[m] g[n - m]
        \label{eq:disc_conv}
    \end{equation}
    Note that in the case that $g$ is non-zero only on a finite interval $[\![-M, M]\!]$ the series in the above definition can be replaced with a finite summation 
    \begin{equation}
        (f \ast g)[n] = \sum_{m=-M}^{M} f[m] g[n - m].
        \label{eq:disc_conv_finite}
    \end{equation}
\end{definition}

The definition of the discrete convolution in Eq. \ref{eq:disc_conv_finite} can be rewritten as a matrix vector multiplication. Write the signal as a vector
$x = [ x[1], \cdots, x[N] ]^{T}$. The convolution with the impulse response $h = [h[1], \cdots, h[M]]^T$ is given by the multiplication with the 
$(M+N-1) \times N$ convolution matrix

\begin{equation}
    y = h \ast x = 
    \begin{bmatrix}
        h[1] & 0 & \cdots & 0 & 0 \\
        h[2] & h[1] & \cdots & 0 & 0 \\
        h[3] & h[2] & \cdots & 0 & 0 \\
        \vdots & h[3] & \cdots & h[1] & 0 \\
        h[m-1] & \vdots & \ddots & h[2] & h[1] \\
        h[m] & h[m-1] &    & \vdots & h[2] \\
        0 & h[m] & \ddots & h[m-2] & \vdots \\
        0 & 0 & \cdots & h[m-1] & h[m-2] \\
        \vdots & \vdots &    & h[m] & h[m-1] \\
        0 & 0 & \cdots & 0 & h[m] 
    \end{bmatrix}
    \begin{bmatrix}
        x[1] \\
        x[2] \\
        x[3] \\
        \vdots \\
        x[N]
    \end{bmatrix}
    \label{eq:conv_matrix}
\end{equation}

This depiction of convolution as matrix transformation is useful as an alternative mathematical perspective on the convolution operation, and it 
was used in the implementation part of this work as calculating the loss gradient over matrix multiplication is already built into the Julia Flux
framework.

\begin{proposition}[Properties of the Convolution operation]
    The following properties hold in both the continuous and the discrete case. For either 
    $f, g, h \in L^1(\mathbb{R}^d)$ or $f, g, h$ complex-valued functions on $\mathbb{Z}$.  

    \begin{enumerate}
            \item Commutativity
                \begin{align*}
                    f \ast g = g \ast f
                \end{align*}
            \item Associativity
                \begin{align*}
                    f \ast (g \ast h) = (f \ast g) \ast h
                \end{align*}
            \item Distributivity
                \begin{align*}
                    f \ast (g + h) = (f \ast g) + (f \ast h)
                \end{align*}
            \item Associativity with scalar multiplication: For $a \in \mathbb{C}$
                \begin{align*}
                    a (f \ast g) = (a f) \ast g
                \end{align*}
            \item Identity: Convolution with the delta distribution $\delta$ is simply the identity operation
                \begin{align*}
                    f \ast \delta = f
                \end{align*}
    \end{enumerate}
    \label{prop:conv_properties}
\end{proposition}

Convolution and the Fourier transform are connected by the following convolutional theorem, which states that the Fourier transform translates 
the convolution operation in the time domain to a simple multiplication in the frequency domain.

\begin{theorem}[Convolutional Theorem]
    Let $f, g \in L^1(\mathbb{R^d})$ be integrable functions on $\mathbb{R^d}$ with Fourier transforms $F=\mathscr{F}\{f\}$ and $G=\mathscr{F}\{g\}$.
    Define their convolution $h = f \ast h$ and its Fourier transform $H = \mathscr{F} \{ f \ast h \}$. The convolutional theorem states that

    \begin{equation}
        H(k) = F(k) \cdot G(k)
    \end{equation}
    where $\cdot$ denotes pointwise multiplication.

    Applying the inverse FT $\mathscr{F}^{-1}$ immediately gives us the corollary
    \begin{equation}
        h(x) = \mathscr{F}^{-1} \{ F \cdot G \}(x).
    \end{equation}

    The convolutional theorem also holds in the discrete case where $f, g$ complex-valued signals and $\mathscr{F}$ denotes the DFT and $\mathscr{F}^{-1}$
    the IDFT.
    \label{th:conv_theorem}
\end{theorem}

The convolutional theorem allows us to efficiently compute convolutions. Note that both the definition in Eq. \ref{eq:disc_conv_finite} and the matrix
multiplication in Eq. \ref{eq:conv_matrix} require $N^2$ operations for $N$ output values. Using the convolutional theorem and the FFT algorithm
the computational complexity can be reduced to $\mathcal{O}(N \log N)$.

If we recall the depiction of the DFT as a unitary transformation in section \ref{sec:unitary_dft} and the convolution matrix in Eq. \ref{eq:conv_matrix}, 
it follows from the convolutional theorem that the DFT diagonalizes convolutional matrices. The Fourier basis is the eigenbasis of the convolution
operation.

\section{Deconvolution}

Deconvolution is the inverse operation to convolution. Given the functions $g$ and $h$, the objective of deconvolution is to find $f$ given the 
convolution equation 

\begin{equation}
    f \ast g = h.
    \label{eq:conv_eq}
\end{equation}

In the simple case of \ref{eq:conv_eq} we can simply use the convolutional theorem \ref{th:conv_theorem} to solve the equation in Fourier space.

\begin{align}
    F(k) G(k) &= H(k) \\
    F(k) &= \frac{H(k)}{G(k)} \\
    f(x) &= \mathscr{F}^{-1}\left(\frac{H}{G}\right)(x) \label{eq:inverse_filtering}
\end{align}

In practice, we however do not have the ideal case shown in \ref{eq:conv_eq}. Usually, $h$ is a recorded signal of a physical system and $f$
is the signal we wish to recover, but that has been convolved by a \textit{filter} or \textit{impulse response} $g$ when measuring the signal.
In real measured systems we typically have a measurement error term that has to be added to \ref{eq:conv_eq}, resulting in 

\begin{equation}
    (f \ast g) + \epsilon = h
    \label{eq:conv_real} 
\end{equation}

In this case $\epsilon$ is the noise term added to our recorded signal. Using the \textit{inverse filtering} approach in Eq. \ref{eq:inverse_filtering}
is now no longer correct. In fact, the signal-noise-ratio is typically worsened and high frequency terms are erroneously amplified. The problems
of deconvolution have long been known in the physics community, a discussion can be found in \cite{jones1970problem}.

\subsection{Wiener Deconvolution} \label{sec:Wiener_deconv}

To address the issues of the simple inverse filtering approach, the so-called Wiener Deconvolution was introduced in \cite{wiener1964extrapolation}. 
This approach again is applied in the frequency domain and attempts to minimize the impact of deconvolved noise at frequencies which have a poor
signal-to-noise ratio.

Given a system as in Eq. \ref{eq:conv_real}, the Wiener deconvolution attempts to provide filter $k$ to estimate $f$ as follows

\begin{equation}
    \hat{f}(t) = (k \ast h)(t)
\end{equation}

where $\hat{f}(t)$ is an estimate of $f(t)$ that minimizes the mean squared error (MSE). The Wiener deconvolution filter is naturally written in 
the Fourier domain as 

\begin{equation}
    K(k) = \frac{G^*(k) S(k)}{|G(k)|^2 S(k) + N(k)}
    \label{eq:wiener_deconv}
\end{equation}

where 

\begin{itemize}
    \item $K(k)$ and $G(k)$ are the Fourier transforms of $k(t)$ and $g(t)$
    \item $S(k) = \mathbb{E}[|F(k)|^2]$ is the mean power spectral density of the original signal $f(t)$
    \item $N(k) = \mathbb{E}[|\mathcal{E}(k)|^2]$ is the mean power spectral density of the noise $\epsilon(t)$
    \item the superscript $^*$ denotes complex conjugation.
\end{itemize}

The filtering operation can then be immediately carried out in the Fourier domain:

\begin{equation}
    \hat{F}(k) = K(k) H(k)
    \label{eq:wiener_filter_fourier}
\end{equation}

Applying the inverse Fourier transform then gives us the estimate of the original signal

\begin{equation}
    \hat{f}(t) = \mathscr{F}^{-1}(\hat{F})(t)
\end{equation}

An important limitation of this approach is that estimates of the exact impulse response of the system
and the mean power spectral density (PSD) of both the noise and the original, unknown signal are needed to perform the deconvolution. In the case of BOLD
time series the impulse response is known, it is the hemodynamic response function. The PSDs of the underlying signal and the noise are however not known.
In the next section on wavelets I will introduce the mathematical tools needed to obtain estimates of the original signal and the noise. 
These will allow us to apply Wiener deconvolution to BOLD time series.
